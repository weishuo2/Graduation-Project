\section{Related Work}
\label{relatework}
Most previous research conducted on graph coloring 
focus on the algorithm design and how to find the chromatic 
number \cite{topc,cor}. FastColor is an iteration-based graph 
coloring algorithm for very large graphs \cite{ijcai-17}. FastColor 
sets the lower bound by finding the clique of the working graph $G_k$, which is reduced from the given graph $G$ and becomes smaller during the 
execution of the algorithm. FastColor sets the upper bound by coloring the $G_m$, which 
collects all the vertices and edges that are removed from the given graph 
$G$. By iteratively coloring the $G_k$ and $G_m$, FastColor can obtain the 
final color plan once the upper bound meets the lower bound. Experimental 
results show that FastColor can color the graph with ten million vertices 
and one hundred million edges in several minutes. 

Focusing on the dynamic graph coloring problem, the work in \cite{vldbcolor} proposed a 
color-propagation based algorithm which colors the vertices according to 
the propagation order of vertices. The authors use a directed acyclic graph as 
the auxiliary graph of the given graph. The authors color the auxiliary graph 
first by exploring all vertices in 2-hop neighbors of the active 
vertices. In the color-propagation based algorithm, the vertices need to be 
recolored only if the in-neighbors have been changed. Experimental results 
show that the algorithm can color the graph with ten million vertices 
and one hundred million edges in seconds. 

The work in \cite{ppopp-11} evaluated the features of graph coloring on GPU, which provides some insightful
research directions for other researchers. Reference \cite{sc-16} studied the graph coloring problem and applied it to the infrastructure-level 
online analytic design space. cuSPARSE library \cite{nvidiaTR} is a basic linear 
algebra subroutines developed by NVIDIA. cuSPARSE contains a set of graph processing tools, including the graph coloring algorithm, which is the only 
publicly-available coloring library for SIMD architectures to date. 
Reference \cite{Manycore} proposed a parallel coloring 
algorithm on manycore architectures and the authors implemented the algorithm 
on the kokkos library \cite{kokkos}, which can be executed on both Xeon Phi and 
GPU with different compile options. Experimental results shows the proposed 
algorithm can achieve up to 1.5$\times$ speed up over cuSPARSE.