\section{Conclusion and Future Opportunities}
\label{conclusion}

This paper focuses on the graph coloring problem on GPU. 
The recursion computation model can cause the long tail 
problem, which leads to the low efficiency, while the 
sequential spread computation model can cause other problems 
such as low convergence speed. This paper propose a high performance 
graph coloring algorithm on GPU, called Feluca. Feluca 
combines the recursion model and the sequential spread model, which significantly improves 
the graph coloring efficiency. We also propose the optimization techniques in Feluca to avoid the atomic operations, including the craftly-designed method to eliminate the cyclic paths in graphs and a top-down coloring 
scheme. Experimental results show that Feluca can achieve up to 
344.58$\times$ speed up over the state-of-art work and at the same time uses fewer colors on some datasets.

In the future, we plan to focus on other aspects of graph 
coloring, such as coloring the graphs in hybrid systems, coloring the 
graphs on new devices and coloring the dynamic graph on GPU and new devices.